% CS70--Spring 2024 --- Homework 02 Solutions
\documentclass[11pt]{article}
\usepackage{amsmath,textcomp,amssymb,geometry,graphicx,enumerate}
\usepackage{xeCJK}

\def\Name{JayZhao}  % Your name
\def\SID{0}  % Your student ID number-
\def\Homework{HW02} % Number of Homework
\def\Session{Spring 2024}

\title{CS70--Spring 2024 --- Homework \Homework \ Solutions}
\author{\Name, SID \SID}
\markboth{CS70--\Session\  Homework \Homework\ \Name}{CS70--\Session\ Homework \Homework\ \Name}
\pagestyle{myheadings}
\date{\today}

\newenvironment{qparts}{\begin{enumerate}[{(}a{)}]}{\end{enumerate}}

\def\endproofmark{$\Box$}
\newenvironment{proof}{\par{\bf 证明}:}{\endproofmark\smallskip}

\textheight=9in
\textwidth=6.5in
\topmargin=-.75in
\oddsidemargin=0.25in
\evensidemargin=0.25in

\begin{document}
\maketitle

% 杂项
\section*{杂项}
\textbf{题目:} 在开始编写最终作业提交之前,请简要说明你是如何完成它的。你还和谁一起合作了?请列出姓名和电子邮件地址。(如果是家庭作业小组,你只需描述一下小组情况。)

我独立完成了本次作业,未与他人合作。

% 1. 通用偏好
\section*{1. 通用偏好}
\textbf{题目:} 假设在稳定匹配实例中偏好是通用的:所有 n 个工作都共享偏好 $C_{1}>C_{2}>\cdots>C_{n}$,所有候选人也共享偏好 $J_{1}>J_{2}>\cdots>J_{n}$。

\begin{qparts}
\item \textbf{题目:} 当由工作方提出配对时,运行算法会得到什么样的配对?你能证明这对所有的 n 都成立吗?
\begin{proof}
\textbf{题意分析:} 所有工作和所有候选人的偏好都是完全一致的,即工作$J_1$最喜欢$C_1$,$J_2$最喜欢$C_2$,以此类推,候选人也同理。

\textbf{算法过程:}
- 第一天,所有工作都向自己最喜欢的候选人提出申请。
- 由于每个候选人也最喜欢向自己提出申请的工作,所以每个候选人都接受了最喜欢的工作。
- 没有拒绝和更换,算法一次完成。

\textbf{例子:} $n=3$时:
- $J_1$向$C_1$申请,$C_1$最喜欢$J_1$,配对成功。
- $J_2$向$C_2$申请,$C_2$最喜欢$J_2$,配对成功。
- $J_3$向$C_3$申请,$C_3$最喜欢$J_3$,配对成功。

\textbf{结论:} 最终配对为$(J_1,C_1),(J_2,C_2),\ldots,(J_n,C_n)$。

\textbf{证明:} 由于偏好完全一致,任何其他配对都会被双方拒绝。假设$J_1$和$C_2$配对,则$J_1$更喜欢$C_1$,$C_2$更喜欢$J_2$,都不会接受。

\textbf{归纳法:} 对$n$成立。
\end{proof}

\item \textbf{题目:} 当由候选人方提出配对时,运行算法会得到什么样的配对?
\begin{proof}
\textbf{算法过程:}
- 所有候选人都向自己最喜欢的工作提出申请。
- 工作也最喜欢这些候选人。
- 所以每个人都和自己排名第一的对象配对。

\textbf{例子:} $n=3$时,$C_1$向$J_1$申请,$J_1$最喜欢$C_1$,配对成功,其他同理。

\textbf{结论:} 最终配对仍然是$(J_1,C_1),(J_2,C_2),\ldots,(J_n,C_n)$。
\end{proof}

\item \textbf{题目:} 这告诉我们关于稳定配对的数量有什么信息?
\begin{proof}
\textbf{结论:} 只有唯一的稳定配对。

\textbf{理由:} 由于所有人的偏好完全一致,任何不同于$(J_1,C_1),(J_2,C_2),\ldots,(J_n,C_n)$的配对都会被某一方拒绝,因此只有唯一的稳定配对。

\textbf{补充说明:} 这说明在极端一致偏好的情况下,稳定配对是唯一的。
\end{proof}
\end{qparts}

% 2. 配对
\section*{2. 配对}
\textbf{题目:} 证明对于每个偶数 $n\ge2$,都存在一个包含 n 个工作和 n 个候选人的稳定匹配问题实例,该实例至少有 $2^{n/2}$ 个不同的稳定匹配。

\begin{proof}
\textbf{题意分析:} 证明存在某种偏好设置,使得稳定配对的数量呈指数级增长。

\textbf{构造方法:}
- 将$n$分成$n/2$组,每组2个工作和2个候选人。
- 每组的偏好如下:
  - $J_{2i-1}$和$J_{2i}$都把$C_{2i-1}$和$C_{2i}$排在最前(顺序可任意),其余随意。
  - $C_{2i-1}$和$C_{2i}$都把$J_{2i-1}$和$J_{2i}$排在最前(顺序可任意),其余随意。

\textbf{每组分析:}
- 每组内部有2种稳定配对方式:$(J_{2i-1},C_{2i-1}),(J_{2i},C_{2i})$ 或 $(J_{2i-1},C_{2i}),(J_{2i},C_{2i-1})$。
- $n/2$组独立选择,总共$2^{n/2}$种稳定配对。

\textbf{例子:} $n=4$时:
- 组1:$J_1,C_1,J_2,C_2$,有2种配对。
- 组2:$J_3,C_3,J_4,C_4$,有2种配对。
- 总共$2\times2=4$种稳定配对。

\textbf{结论:} 通过分组构造,稳定配对数至少为$2^{n/2}$。
\end{proof}

% 3. 上限
\section*{3. 上限}
\textbf{题目:}
(a) 在笔记中,我们证明了稳定匹配算法最多在 $n^{2}$ 天内终止。请证明以下更强的结论:稳定匹配算法将总是在最多 $(n-1)^{2}+1=n^{2}-2n+2$ 天内终止。

(b) 为4个工作和4个候选人提供一组偏好列表,当运行"提议-拒绝"算法时,将达到(a)部分中的上限。通过在你的偏好列表上运行"提议-拒绝"算法来验证这一点。

\begin{qparts}
\item \textbf{题目:} 在笔记中,我们证明了稳定匹配算法最多在 $n^{2}$ 天内终止。请证明以下更强的结论:稳定匹配算法将总是在最多 $(n-1)^{2}+1=n^{2}-2n+2$ 天内终止。
\begin{proof}
\textbf{题意分析:} 证明比$n^2$更紧的上界。

\textbf{分析:}
- 每个工作最多被拒绝$n-1$次(因为每个候选人最多拒绝$n-1$次)。
- 每次被拒绝后只能向下一个候选人申请。
- 最多有$n$个工作,每个工作最多提出$n$次申请(但最后一次不会被拒绝)。

\textbf{最坏情况:}
- 前$n-1$天,每天都有一个工作被拒绝。
- 最后一天所有人配对成功。

\textbf{计算:}
- 每个工作最多被拒绝$n-1$次,$n$个工作共$(n-1)n$次拒绝。
- 加上最后一天配对成功,总共$(n-1)n+1=(n-1)^2+1$天。

\textbf{结论:} 算法最多$(n-1)^2+1$天终止。
\end{proof}

\item 给出4个工作和4个候选人的偏好列表,使得算法达到上限,并验证。
\begin{proof}
\textbf{构造:}
- 工作$J_1$偏好:$C_1>C_2>C_3>C_4$
- 工作$J_2$偏好:$C_2>C_3>C_4>C_1$
- 工作$J_3$偏好:$C_3>C_4>C_1>C_2$
- 工作$J_4$偏好:$C_4>C_1>C_2>C_3$

- 候选人$C_1$偏好:$J_2>J_3>J_4>J_1$
- 候选人$C_2$偏好:$J_3>J_4>J_1>J_2$
- 候选人$C_3$偏好:$J_4>J_1>J_2>J_3$
- 候选人$C_4$偏好:$J_1>J_2>J_3>J_4$

\textbf{算法过程:}
- 第一天,所有工作向最喜欢的候选人申请,但都被拒绝(因为候选人更喜欢其他人)。
- 每天只有一个配对成功,其余被拒绝。
- 直到最后一天所有人配对成功。

\textbf{验证:} 共$(4-1)^2+1=10$天。

\textbf{补充说明:} 这种构造使得每一步都尽量拖延配对,达到上限。
\end{proof}
\end{qparts}

% 4. 逐步建立错误?
\section*{4. 逐步建立错误?}
\textbf{题目:}
下面的"证明"有什么问题?除了找到一个反例,你还应该解释这种方法的根本性错误是什么,以及为什么它揭示了逐步建立错误的危险。

\textbf{错误论断}:如果一个无向图中的每个顶点度数至少为1,则该图是连通的。

\textbf{证明?} 我们对顶点数 $n\ge1$ 使用归纳法。

\textbf{基础情形}:只有一个顶点的图只有一个,其度数为0。因此,基础情形是无意义地成立的,因为如果部分是假的。

\textbf{归纳假设}:假设该论断对于某个 $n\ge1$ 成立。

\textbf{归纳步骤}:我们证明该论断对 $n+1$ 也成立。考虑一个有 n 个顶点的无向图,其中每个顶点的度数至少为1。根据归纳假设,该图是连通的。现在添加一个顶点 x,得到一个有 $(n+1)$ 个顶点的图,如下所示。

\begin{center}
\includegraphics[width=0.4\linewidth]{week2/image3.png}
\end{center}

剩下的就是要检查是否存在从 x 到每个其他顶点 z 的路径。由于 x 的度数至少为1,存在一条从 x 到某个其他顶点(称之为 y)的边。因此,我们可以通过将边 {x,y} 连接到从 y 到 z 的路径上来获得从 x 到 z 的路径。这证明了该论断对 $n+1$ 成立。

\begin{proof}
\textbf{反例:}
考虑两个不连通的分量,每个分量都是一条边。例如,顶点集$\{a,b,c,d\}$,边集$\{(a,b),(c,d)\}$。每个顶点度数都为1,但图不连通。

\textbf{归纳法的根本性错误:}
归纳假设认为"对所有$n$个顶点的图,每个顶点度数至少为1就连通",但实际上$n$个顶点的图可能有多个连通分量。归纳步骤中"加一个顶点$x$"时,$x$只需与某个分量中的一个顶点相连即可满足度数条件,但无法保证$x$与所有原有顶点连通。

\textbf{逐步建立的危险:}
归纳法适用于"局部到整体"性质时要小心。这里的错误在于,局部满足度数条件并不能推出全局连通性。归纳法不能保证新加的顶点与所有原有顶点连通,可能只与某个分量连通,导致整体不连通。

\textbf{总结:} 该证明忽略了图可能有多个分量,归纳法不能用于证明"全局"性质时要格外小心。
\end{proof}

% 5. 图中的证明
\section*{5. 图中的证明}
\textbf{题目:}
(a) 在从旧金山交通习惯到洛杉矶交通习惯的坐标轴上,旧加州更偏向旧金山:也就是说,更文明。在旧加州,所有道路都是单行道。假设旧加州有 n 个城市 $(n\ge2)$,对于每对城市 X 和 Y,要么有从 X 到 Y 的道路,要么有从 Y 到 X 的道路。

证明存在一个城市,从任何其他城市出发,通过最多2条道路即可到达。[提示:归纳法]

(b) 考虑一个有 n 个顶点的连通图 G,它恰好有 2m 个奇数度的顶点,其中 $m>0$。证明存在 m 条路径,这些路径共同覆盖了 G 的所有边(即,G 的每条边恰好出现在 m 条路径中的一条中,并且每条路径不应多次包含任何特定的边)。

[提示:在讲座中,我们已经证明了一个连通无向图有欧拉回路当且仅当每个顶点的度数都是偶数。这个事实在证明中可能有用。]

(c) 证明任何图 G 是二分图当且仅当它没有奇数长度的环路。[提示:在其中一个证明方向,考虑从一个给定顶点出发的路径长度。]

\begin{qparts}
\item \textbf{题目:} 证明存在一个城市,从任何其他城市出发,通过最多2条道路即可到达。
\begin{proof}
\textbf{详细证明:}

我们用归纳法证明。  
\textbf{基础情形}:$n=2$,有两个城市$A,B$,要么$A\to B$,要么$B\to A$,任选一个城市即可满足条件。

\textbf{归纳假设}:假设对于$n$个城市,存在一个城市$C$,从任意其他城市出发,最多经过2条路就能到达$C$。

\textbf{归纳步骤}:现在增加一个新城市$X$,总共有$n+1$个城市。对于每个原城市$Y$,要么有$X\to Y$,要么$Y\to X$。

- 如果$X$能直接到达所有原城市(即$X\to Y$对所有$Y$成立),那么$X$就是所求城市。
- 否则,存在某个$Y$使$Y\to X$。由归纳假设,存在城市$C$,任意原城市$Z$都能在2步内到达$C$。
  - 如果$C=X$,前面已讨论。
  - 如果$C\neq X$,那么对于$X$以外的任意城市$Z$,$Z$能在2步内到达$C$。对于$X$,如果$X\to C$,则$X$也能1步到$C$;如果$C\to X$,则存在$Y$使$Y\to X$,而$Y$能2步到$C$,所以$X$能通过$Y$和$C$,2步到$C$。

\textbf{结论:} 总能找到这样的城市$C$,使得任意城市最多2步到达$C$。
\end{proof}

\item \textbf{题目:} 证明存在m条路径覆盖所有边,每条路径不重复边。
\begin{proof}
\textbf{详细证明:}

已知$G$是连通图,有$2m$个奇数度顶点。我们要证明:可以用$m$条不重复边的路径覆盖所有边。

\textbf{步骤:}
1. 将$2m$个奇数度顶点两两配对,假设配对为$(v_1,w_1), (v_2,w_2), ..., (v_m,w_m)$。
2. 对于每一对$(v_i,w_i)$,在$G$中添加一条"虚拟边"$v_i-w_i$,得到新图$G'$。
3. 这样$G'$中所有顶点度数都为偶数,所以$G'$存在欧拉回路(每条边恰好走一次的回路)。
4. 在欧拉回路中,每遇到一条虚拟边就断开,把欧拉回路分割成$m$条路径,每条路径的起点和终点分别是一对奇数度顶点。
5. 这些路径覆盖了$G$的所有边,且每条边只出现一次。

\textbf{例子:}  
假设有4个奇数度点$a,b,c,d$,配对为$(a,b),(c,d)$,加两条虚拟边后欧拉回路断成2条路径,正好覆盖所有边。

\textbf{结论:} $G$的所有边可以被$m$条不重复的路径覆盖。
\end{proof}

\item \textbf{题目:} 证明G是二分图当且仅当没有奇数长度环。
\begin{proof}
\textbf{详细证明:}

\textbf{必要性(有奇环则不是二分图):}  
假设$G$是二分图,顶点集可分为$A,B$两组,所有边都在$A$和$B$之间。若存在奇数长度环$C_{2k+1}$,环上顶点交替分组,回到起点时会出现同组相邻,矛盾。

\textbf{充分性(无奇环则是二分图):}  
假设$G$没有奇数环。任选一个顶点$v$,以$v$为根做BFS,按距离奇偶将顶点分为两组$A,B$。若有一条边连接同组顶点,则存在一条从$v$到这两个顶点的路径,加上这条边形成奇数环,矛盾。

\textbf{例子:}  
- $C_3$(三角形)不是二分图,因为有奇数环。
- $C_4$(四边形)是二分图,没有奇数环。

\textbf{结论:} $G$是二分图$\iff$没有奇数长度环。
\end{proof}
\end{qparts}

% 6. (可选) 无胜于有
\section*{6. (可选) 无胜于有}
\textbf{题目:}
在稳定匹配问题中,假设一些工作和候选人有硬性要求,可能无法将就于任何配对。换句话说,每个工作/候选人宁愿不匹配,也不愿与偏好列表中低于某个特定点的对象匹配。我们用"实体"一词来指代候选人/工作。一个匹配最终可能是部分的,即一些实体会且应该保持不匹配。因此,这里的稳定性概念需要稍作调整,以体现工作方单方面解雇员工和/或员工自行离开的自主权。

一个匹配是\textbf{稳定}的,如果
\newline
* 不存在已匹配的实体,宁愿不匹配也不愿与当前伴侣在一起;\newline
* 不存在已匹配/已满员的工作和未匹配的候选人,他们都宁愿与对方匹配,而不是维持现状;\newline
* 不存在已匹配的工作和已匹配的候选人,他们都宁愿与对方匹配,而不是与当前伴侣在一起;\newline
* 以及,同样地,不存在未匹配的工作和已匹配的候选人,他们都宁愿与对方匹配,而不是维持现状;\newline
* 不存在未匹配的工作和未匹配的候选人,他们都宁愿与对方匹配,也不愿保持不匹配状态。

(a) 证明在允许实体不匹配的情况下,稳定的配对仍然存在。(提示:你可以通过引入虚拟实体来解决这个问题,当工作/候选人未匹配时,他们就与这些虚拟实体"匹配"。你应该如何调整工作/候选人的偏好列表,包括新引入的虚拟实体的偏好列表,以使其奏效?)

(b) 正如你在讲座中看到的,我们可能会有不同的稳定匹配。但有趣的是,如果一个实体在一个稳定匹配中未匹配,那么在任何其他稳定匹配中它也必须保持未匹配。请用反证法证明这一事实。

\begin{qparts}
\item \textbf{题目:} 证明允许实体不匹配时,稳定配对仍然存在。
\begin{proof}
\textbf{什么是Gale-Shapley算法?}  
Gale-Shapley算法(又称"提议-拒绝"算法)是一种经典的稳定婚姻匹配算法。它的基本思想是:一方(比如工作)主动向自己最喜欢的对象(比如候选人)提出申请,被申请方根据自己的偏好选择是否接受,没被选中的申请人继续向下一个喜欢的对象申请。这个过程不断进行,直到没有人想再提出申请为止。最终得到的配对是"稳定"的,即不会有一对互相更喜欢对方而抛弃现有配对的情况。

\textbf{本题的难点:}  
有些工作或候选人宁愿单身(不匹配),也不愿和某些对象配对。比如某公司宁愿空缺,也不想招一个很不喜欢的员工。

\textbf{如何处理?}  
我们可以为每个"宁愿单身"的实体引入一个"虚拟对象"(dummy),比如"虚拟候选人"或"虚拟工作"。  
- 如果某工作$J$的偏好表中,$C_5$以下的候选人都不想要,就把"虚拟候选人"排在$C_5$之后。
- $J$如果没有匹配到$C_1$到$C_5$,就会和虚拟候选人"配对",表示$J$选择空缺。
- 虚拟对象的偏好表可以随意,只要保证不会和真实对象形成不稳定对即可。

\textbf{算法过程:}  
1. 把所有虚拟对象加入到偏好表中。
2. 用Gale-Shapley算法运行配对。
3. 如果某个实体最终和虚拟对象配对,说明它选择了单身。

\textbf{为什么这样一定有稳定配对?}  
- Gale-Shapley算法本身就能保证存在稳定配对。
- 虚拟对象的引入只是把"单身"变成了"和虚拟对象配对",不会破坏算法的稳定性和收敛性。

\textbf{生活例子:}  
比如有4家公司和4位求职者,但有1家公司宁愿空缺也不想招某些人。我们就给这家公司加一个"虚拟求职者",让它在不满意时选择"空缺"。算法运行后,所有人要么配对,要么单身(和虚拟对象配对),且不会出现"互相更喜欢对方却没配对"的不稳定情况。

\textbf{结论:}  
允许实体不匹配时,稳定配对依然存在。
\end{proof}

\item \textbf{题目:} 反证法证明:若某实体在一个稳定配对中未匹配,则在所有稳定配对中都未匹配。
\begin{proof}
\textbf{通俗解释:}  
假设有一个人$A$在某个稳定配对$M_1$中是单身(和虚拟对象配对),但在另一个稳定配对$M_2$中和$B$配对。  
- 在$M_1$中,$B$要么也单身,要么和别人配对。
- 如果$A$和$B$在$M_2$中能配对,说明他们彼此都更喜欢对方而不是单身或现有配对。
- 这就意味着在$M_1$中,$A$和$B$会形成"不稳定对",因为他们都更喜欢和对方在一起,而不是各自的现状。
- 这和$M_1$是稳定配对矛盾。

\textbf{结论:}  
如果某实体在一个稳定配对中未匹配(单身),那么在所有稳定配对中都未匹配。
\end{proof}
\end{qparts}

\end{document}
