% Search for all the places that say "PUT SOMETHING HERE".

\documentclass[11pt]{article}
\usepackage{amsmath,textcomp,amssymb,geometry,graphicx,enumerate}
\usepackage{xeCJK}


\def\Name{吴彦祖}  % Your name
\def\SID{20211003341}  % Your student ID number-
\def\Homework{0A} % Number of Homework
\def\Session{Spring 2024}

\title{CS70--Spring 2024 --- Discussion \Homework \  Solutions}
\author{\Name, SID \SID}
\markboth{CS70--\Session\  Discussion \Homework\ \Name}{CS70--\Session\ Discussion \Homework\ \Name}
\pagestyle{myheadings}
\date{\today}

\newenvironment{qparts}{\begin{enumerate}[{(}a{)}]}{\end{enumerate}}
\def\endproofmark{$\Box$}
\newenvironment{proof}{\par{\bf Proof}:}{\endproofmark\smallskip}

\textheight=9in
\textwidth=6.5in
\topmargin=-.75in
\oddsidemargin=0.25in
\evensidemargin=0.25in


\begin{document}
\maketitle

Collaborators: 吴彦祖

\section*{1. 命题练习}
将以下句子转化为命题逻辑,并将以下命题转化为中文。说明每个语句是否为真,并给出简要理由。

\begin{qparts}
\item 存在一个不是有理数的实数。

命题逻辑表达:$\exists x \in \mathbb{R},\ x \notin \mathbb{Q}$\\
真。理由:如 $\sqrt{2}$ 就是一个无理数。

\item 所有整数要么是自然数,要么是负数,但不能两者兼是。

命题逻辑表达:$\forall x \in \mathbb{Z},\ (x \in \mathbb{N}) \vee (x < 0)$,且$(x \in \mathbb{N}) \wedge (x < 0)$ 不成立。\\
真。理由:自然数和负数没有交集,所有整数要么属于自然数(通常指非负整数),要么是负数。

\item 如果一个自然数能被6整除,那么它能被2整除或能被3整除。

命题逻辑表达:$\forall x \in \mathbb{N},\ 6|x \implies (2|x \vee 3|x)$\\
真。理由:6的倍数必然是2和3的倍数之一。

\item $(\forall x \in \mathbb{Z})(x \in \mathbb{Q})$

中文翻译:所有整数都是有理数。\\
真。理由:每个整数都可以写成分数形式(如 $n = \frac{n}{1}$)。

\item $(\forall x \in \mathbb{Z})(((2 \mid x) \vee (3 \mid x)) \Longrightarrow (6 \mid x))$

中文翻译:所有整数中,只要能被2或3整除,就能被6整除。\\
假。理由:反例:$x=2$ 能被2整除但不能被6整除。

\item $(\forall x \in \mathbb{N})(x > 7 \Longrightarrow (\exists a, b \in \mathbb{N})(a + b = x))$

中文翻译:所有大于7的自然数都可以表示为两个自然数之和。\\
真。理由:取 $a=1, b=x-1$ 即可。

\end{qparts}

\section*{2. 真值表}
通过写出真值表,判断以下等价关系是否成立。明确说明每对是否等价。

\begin{qparts}
\item $P \wedge (Q \vee P) \equiv P \wedge Q$

\begin{center}
\begin{tabular}{|c|c|c|c|c|}
\hline
$P$ & $Q$ & $Q \vee P$ & $P \wedge (Q \vee P)$ & $P \wedge Q$ \\
\hline
T & T & T & T & T \\
T & F & T & T & F \\
F & T & T & F & F \\
F & F & F & F & F \\
\hline
\end{tabular}
\end{center}

不等价。理由:当 $P$ 为真,$Q$ 为假时,左边为真,右边为假。

\item $(P \vee Q) \wedge R \equiv (P \wedge R) \vee (Q \wedge R)$

\begin{center}
\begin{tabular}{|c|c|c|c|c|c|c|c|}
\hline
$P$ & $Q$ & $R$ & $P \vee Q$ & $(P \vee Q) \wedge R$ & $P \wedge R$ & $Q \wedge R$ & $(P \wedge R) \vee (Q \wedge R)$ \\
\hline
T & T & T & T & T & T & T & T \\
T & T & F & T & F & F & F & F \\
T & F & T & T & T & T & F & T \\
T & F & F & T & F & F & F & F \\
F & T & T & T & T & F & T & T \\
F & T & F & T & F & F & F & F \\
F & F & T & F & F & F & F & F \\
F & F & F & F & F & F & F & F \\
\hline
\end{tabular}
\end{center}

等价。理由:分配律,两边恒等。

\item $(P \wedge Q) \vee R \equiv (P \vee R) \wedge (Q \vee R)$

\begin{center}
\begin{tabular}{|c|c|c|c|c|c|c|c|}
\hline
$P$ & $Q$ & $R$ & $P \wedge Q$ & $(P \wedge Q) \vee R$ & $P \vee R$ & $Q \vee R$ & $(P \vee R) \wedge (Q \vee R)$ \\
\hline
T & T & T & T & T & T & T & T \\
T & T & F & T & T & T & T & T \\
T & F & T & F & T & T & T & T \\
T & F & F & F & F & T & F & F \\
F & T & T & F & T & T & T & T \\
F & T & F & F & F & F & T & F \\
F & F & T & F & T & T & T & T \\
F & F & F & F & F & F & F & F \\
\hline
\end{tabular}
\end{center}

等价。理由:吸收律,两边恒等。

\end{qparts}

\section*{3. 蕴含}
以下哪些蕴含无论 $P$ 取何值总是为真?对于每个假的断言,给出一个反例(即找出一个 $P(x, y)$ 使得蕴含为假)。

\begin{qparts}
\item $\forall x \forall y P(x, y) \Longrightarrow \forall y \forall x P(x, y)$

恒真。理由:量词顺序互换不影响全称命题。

\item $\forall x \exists y P(x, y) \Longrightarrow \exists y \forall x P(x, y)$

不恒真。反例:设 $P(x, y)$ 表示“$x < y$”,则对每个 $x$ 存在 $y$ 使得 $x < y$,但不存在某个 $y$ 使得对所有 $x$ 都有 $x < y$。

\item $\exists x \forall y P(x, y) \Longrightarrow \forall y \exists x P(x, y)$

恒真。理由:如果存在某个 $x$ 对所有 $y$ 都成立,则对每个 $y$ 取这个 $x$ 即可。

\end{qparts}

\end{document}
