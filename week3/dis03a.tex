% CS70--Spring 2024 --- Discussion 3A Solutions
\documentclass[11pt]{article}
\usepackage{amsmath,textcomp,amssymb,geometry,graphicx,enumerate}
\usepackage{xeCJK}

\def\Name{JayZhao} % Your name
\def\SID{0} % Your student ID number-
\def\Homework{3A} % Number of Homework
\def\Session{Spring 2024}

\title{CS70--Spring 2024 --- Discussion \Homework \ Solutions}
\author{\Name, SID \SID}
\markboth{CS70--\Session\ Discussion \Homework\ \Name}{CS70--\Session\ Discussion \Homework\ \Name}
\pagestyle{myheadings}
\date{\today}

\newenvironment{qparts}{\begin{enumerate}[{(}a{)}]}{\end{enumerate}}
\def\endproofmark{$\blacksquare$}
\newenvironment{proof}{\par\noindent{\bf 证明}:}{\endproofmark\smallskip}

\textheight=9in
\textwidth=6.5in
\topmargin=-.75in
\oddsidemargin=0.25in
\evensidemargin=0.25in

\begin{document}
\maketitle

% 1. 派对技巧
\section*{1. 派对技巧}
\begin{qparts}
\item 求 $113^{142}$ 的最后一位数字。

\begin{proof}
我们要求解 $113^{142}$ 的最后一位数字,这等价于计算 $113^{142} \pmod{10}$ 的值。

首先,根据模运算的性质,如果 $a \equiv b \pmod{m}$,则 $a^k \equiv b^k \pmod{m}$。
因为 $113 = 110 + 3$,所以 $113 \equiv 3 \pmod{10}$。
因此,$113^{142} \equiv 3^{142} \pmod{10}$。

接下来,我们考察 $3$ 的幂次模 $10$ 的循环规律:
\begin{itemize}
    \item $3^1 \equiv 3 \pmod{10}$
    \item $3^2 \equiv 9 \pmod{10}$
    \item $3^3 \equiv 27 \equiv 7 \pmod{10}$
    \item $3^4 \equiv 3 \times 7 = 21 \equiv 1 \pmod{10}$
    \item $3^5 \equiv 3 \times 1 = 3 \pmod{10}$
\end{itemize}
我们发现 $3$ 的幂次模 $10$ 的结果以 $4$ 为周期循环($3, 9, 7, 1$)。

为了确定 $3^{142}$ 在这个周期中的位置,我们用指数 $142$ 除以周期长度 $4$:
\[ 142 = 4 \times 35 + 2 \]
所以,$142 \equiv 2 \pmod{4}$。
这意味着 $3^{142}$ 的个位数与 $3^2$ 的个位数相同。
\[ 3^{142} = 3^{4 \times 35 + 2} = (3^4)^{35} \cdot 3^2 \equiv 1^{35} \cdot 9 \equiv 9 \pmod{10} \]
因此,$113^{142}$ 的最后一位数字是 $9$。
\end{proof}

\item 求 $9^{99999}$ 的最后一位数字。

\begin{proof}
我们需求解 $9^{99999} \pmod{10}$。
考察 $9$ 的幂次模 $10$ 的规律:
\begin{itemize}
    \item $9^1 \equiv 9 \pmod{10}$
    \item $9^2 \equiv 81 \equiv 1 \pmod{10}$
    \item $9^3 \equiv 9 \times 1 \equiv 9 \pmod{10}$
\end{itemize}
$9$ 的幂次模 $10$ 的结果以 $2$ 为周期循环($9, 1$)。当指数为奇数时,结果为 $9$;当指数为偶数时,结果为 $1$。

由于指数 $99999$ 是一个奇数,我们可以将其写成 $2k+1$ 的形式(其中 $k=49999$)。
\[ 9^{99999} = 9^{2k+1} = (9^2)^k \cdot 9^1 \equiv 1^k \cdot 9 \equiv 9 \pmod{10} \]
因此,$9^{99999}$ 的最后一位数字是 $9$。
\end{proof}

\item 求 $36^{41}$ 的最后一位数字。

\begin{proof}
我们需求解 $36^{41} \pmod{10}$。
首先,$36 \equiv 6 \pmod{10}$,所以 $36^{41} \equiv 6^{41} \pmod{10}$。

我们观察到 $6$ 的任何正整数次幂的个位数都是 $6$。我们可以用数学归纳法来证明对于所有正整数 $n \ge 1$,$6^n \equiv 6 \pmod{10}$。
\begin{itemize}
    \item \textbf{基础情形}: 当 $n=1$ 时,$6^1 = 6 \equiv 6 \pmod{10}$。命题成立。
    \item \textbf{归纳假设}: 假设对于某个整数 $k \ge 1$,$6^k \equiv 6 \pmod{10}$ 成立。
    \item \textbf{归纳步骤}: 我们需要证明 $6^{k+1} \equiv 6 \pmod{10}$。
    \[ 6^{k+1} = 6^k \cdot 6 \]
    根据归纳假设,$6^k \equiv 6 \pmod{10}$。代入上式得:
    \[ 6^{k+1} \equiv 6 \cdot 6 = 36 \equiv 6 \pmod{10} \]
\end{itemize}
归纳得证。由于对于所有 $n \ge 1$,$6^n \equiv 6 \pmod{10}$,那么当 $n=41$ 时,该结论也成立。
因此,$36^{41}$ 的最后一位数字是 $6$。
\end{proof}
\end{qparts}

\newpage

% 2. 模运算杂烩
\section*{2. 模运算杂烩}
\begin{qparts}
\item 存在某个整数 $x\in\mathbb{Z}$,使得 $x\equiv3$ (mod 16) 且 $x\equiv4$ (mod 6) 吗?

\begin{proof}
不存在这样的整数 $x$。我们可以从奇偶性来分析:
\begin{enumerate}
    \item 第一个同余式 $x \equiv 3 \pmod{16}$ 意味着 $x = 16k + 3$ 对于某个整数 $k$ 成立。由于 $16k$ 是偶数,而 $3$ 是奇数,所以 $x$ 必须是一个\textbf{奇数}。
    \item 第二个同余式 $x \equiv 4 \pmod{6}$ 意味着 $x = 6j + 4$ 对于某个整数 $j$ 成立。由于 $6j$ 是偶数,而 $4$ 也是偶数,所以 $x$ 必须是一个\textbf{偶数}。
\end{enumerate}
一个整数不可能同时是奇数和偶数。这个矛盾说明满足这两个条件的整数 $x$ 不存在。

\textbf{注}:中国剩余定理要求模数两两互质。这里 $\text{gcd}(16, 6) = 2 \neq 1$,不满足定理的充分条件,但这并不直接意味着无解。只有当条件之间产生矛盾时,才确定无解。
\end{proof}

\item 若 $2x\equiv4$ (mod 12),则 $x\equiv2$ (mod 12) 吗?

\begin{proof}
该结论\textbf{不成立}。

在模运算中,形如 $ax \equiv ab \pmod m$ 的同余式,只有当 $\text{gcd}(a, m) = 1$ 时,我们才能安全地“约去” $a$ 得到 $x \equiv b \pmod m$。

在本例中,$2x \equiv 4 \pmod{12}$,而 $\text{gcd}(2, 12) = 2 \neq 1$,所以我们不能直接两边除以 $2$。

为了说明该结论错误,我们只需提供一个反例。
令 $x=8$。
$2 \times 8 = 16$,而 $16 \equiv 4 \pmod{12}$。所以 $x=8$ 是同余方程 $2x \equiv 4 \pmod{12}$ 的一个解。
然而,$8 \not\equiv 2 \pmod{12}$。
因此,从 $2x\equiv4$ (mod 12) 并不能推出 $x\equiv2$ (mod 12)。
\end{proof}

\item 若 $2x\equiv4$ (mod 12),则 $x\equiv2$ (mod 6) 吗?

\begin{proof}
该结论\textbf{成立}。

根据同余的定义,$a \equiv b \pmod m$ 等价于 $m \mid (a-b)$,即 $a-b$ 是 $m$ 的整数倍。
$2x \equiv 4 \pmod{12}$ 意味着 $12 \mid (2x - 4)$。
这意味着存在某个整数 $k$,使得:
\[ 2x - 4 = 12k \]
将整个等式两边同时除以 $2$(这是在整数范围内进行的,是允许的):
\[ x - 2 = 6k \]
根据同余的定义,这个等式 $x-2=6k$ 恰好意味着 $6 \mid (x-2)$。
因此,$x \equiv 2 \pmod{6}$。
\end{proof}
\end{qparts}

\newpage

% 3. 模逆元
\section*{3. 模逆元}
\begin{qparts}
\item 3 是 5 模 10 的逆元吗?

\begin{proof}
\textbf{不是}。一个数 $a$ 存在模 $m$ 的逆元,当且仅当 $\text{gcd}(a, m) = 1$。
在这里,我们需要判断 $5$ 是否存在模 $10$ 的逆元。
计算最大公约数:$\text{gcd}(5, 10) = 5 \neq 1$。
因为 $5$ 和 $10$ 不互质,所以 $5$ 模 $10$ 不存在乘法逆元。因此,任何数(包括 $3$)都不可能是它的逆元。

或者,直接通过定义验证:
$5 \times 3 = 15 \equiv 5 \pmod{10}$。
由于结果不等于 $1$,所以 $3$ 不是 $5$ 模 $10$ 的逆元。
\end{proof}

\item 3 是 5 模 14 的逆元吗?

\begin{proof}
\textbf{是}。根据模逆元的定义,我们需要验证 $5 \times 3 \equiv 1 \pmod{14}$ 是否成立。
计算可得:
\[ 5 \times 3 = 15 \]
而 $15 = 14 + 1$,所以 $15 \equiv 1 \pmod{14}$。
因此,$3$ 是 $5$ 模 $14$ 的逆元。
\end{proof}

\item 对于所有 $n\in\mathbb{N}$,$3+14n$ 是 5 模 14 的逆元吗?

\begin{proof}
\textbf{是}。我们需要验证对于任意自然数 $n$,$5 \times (3+14n) \equiv 1 \pmod{14}$ 是否成立。
展开左侧表达式:
\[ 5 \times (3+14n) = 15 + 70n \]
现在我们对该结果模 $14$:
\[ 15 + 70n \pmod{14} \]
由于 $15 \equiv 1 \pmod{14}$ 且 $70n = (14 \times 5)n \equiv 0 \pmod{14}$,我们有:
\[ 15 + 70n \equiv 1 + 0 \equiv 1 \pmod{14} \]
因此,对于所有 $n\in\mathbb{N}$,$3+14n$ 都是 $5$ 模 $14$ 的逆元。这表明逆元在模 $m$ 的意义下是唯一的,但具体表示形式可以有无穷多种。
\end{proof}

\item 4 模 8 有逆元吗?

\begin{proof}
\textbf{没有}。一个数 $a$ 存在模 $m$ 的乘法逆元,当且仅当 $a$ 和 $m$ 互质,即 $\text{gcd}(a, m) = 1$。
在本例中,$a=4$,$m=8$。
\[ \text{gcd}(4, 8) = 4 \]
由于 $\text{gcd}(4, 8) \neq 1$,$4$ 和 $8$ 不互质,所以 $4$ 模 $8$ 不存在乘法逆元。
\end{proof}

\item 假设 $x,x^{\prime}\in\mathbb{Z}$ 都是 a 模 m 的逆元。是否可能 $x\not\equiv x^{\prime}$ (mod m)?

\begin{proof}
\textbf{不可能}。模 $m$ 的逆元(如果存在)在模 $m$ 的意义下是唯一的。

假设 $x$ 和 $x'$ 都是 $a$ 模 $m$ 的逆元。根据定义,我们有:
\begin{enumerate}
    \item $ax \equiv 1 \pmod{m}$
    \item $ax' \equiv 1 \pmod{m}$
\end{enumerate}
由这两个同余式可得:
\[ ax \equiv ax' \pmod{m} \]
移项可得:
\[ ax - ax' \equiv 0 \pmod{m} \]
\[ a(x - x') \equiv 0 \pmod{m} \]
因为 $a$ 存在模 $m$ 的逆元,我们知道 $\text{gcd}(a, m) = 1$。这意味着我们可以用 $a$ 的逆元(记作 $a^{-1}$)乘以同余式两边:
\[ a^{-1} \cdot a(x - x') \equiv a^{-1} \cdot 0 \pmod{m} \]
\[ 1 \cdot (x - x') \equiv 0 \pmod{m} \]
\[ x - x' \equiv 0 \pmod{m} \]
这等价于 $x \equiv x' \pmod{m}$。
因此,任意两个 $a$ 模 $m$ 的逆元在模 $m$ 的意义下必然是相等的。
\end{proof}
\end{qparts}

\newpage

% 4. 斐波那契最大公约数
\section*{4. 斐波那契最大公约数}
\begin{proof}
我们将使用数学归纳法证明,对于所有 $n \ge 1$,斐波那契数 $F_n$ 和 $F_{n-1}$ 互质,即 $\text{gcd}(F_n, F_{n-1}) = 1$。(斐波那契数列定义为 $F_0=0, F_1=1, F_n = F_{n-1} + F_{n-2}$ for $n \ge 2$)。

\textbf{基础情形}:
对于 $n=1$,我们需要计算 $\text{gcd}(F_1, F_0)$。
$\text{gcd}(F_1, F_0) = \text{gcd}(1, 0) = 1$。
对于 $n=2$,我们需要计算 $\text{gcd}(F_2, F_1)$。
$F_2 = F_1 + F_0 = 1+0=1$。$\text{gcd}(F_2, F_1) = \text{gcd}(1, 1) = 1$。
基础情形成立。

\textbf{归纳假设}:
假设对于某个整数 $k \ge 1$,$\text{gcd}(F_k, F_{k-1}) = 1$ 成立。

\textbf{归纳步骤}:
我们需要证明 $\text{gcd}(F_{k+1}, F_k) = 1$。
根据斐波那契数列的定义,$F_{k+1} = F_k + F_{k-1}$。
我们将此代入最大公约数的计算中:
\[ \text{gcd}(F_{k+1}, F_k) = \text{gcd}(F_k + F_{k-1}, F_k) \]
根据欧几里得算法的一个基本性质:对于任意整数 $a, b$,$\text{gcd}(a, b) = \text{gcd}(a-b, b)$。反复应用该性质可得 $\text{gcd}(a, b) = \text{gcd}(a \pmod b, b)$。在这里,$\text{gcd}(F_k + F_{k-1}, F_k) = \text{gcd}((F_k + F_{k-1}) - F_k, F_k) = \text{gcd}(F_{k-1}, F_k)$。
所以,我们有:
\[ \text{gcd}(F_{k+1}, F_k) = \text{gcd}(F_k, F_{k-1}) \]
根据我们的归纳假设,$\text{gcd}(F_k, F_{k-1}) = 1$。
因此,$\text{gcd}(F_{k+1}, F_k) = 1$。

由数学归纳法原理,该命题对于所有 $n \ge 1$ 均成立。
\end{proof}

\end{document}