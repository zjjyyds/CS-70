\documentclass[11pt]{article}
\usepackage{amsmath,textcomp,amssymb,geometry,graphicx,enumerate}
\usepackage{xeCJK}

\def\Name{JayZhao}  % Your name
\def\SID{0}  % Your student ID number-
\def\Homework{1B} % Number of Homework
\def\Session{Spring 2024}

\title{CS70--Spring 2024 --- Discussion \Homework \ Solutions}
\author{\Name, SID \SID}
\markboth{CS70--\Session\ Discussion \Homework\ \Name}{CS70--\Session\ Discussion \Homework\ \Name}
\pagestyle{myheadings}
\date{\today}

\newenvironment{qparts}{\begin{enumerate}[{(}a{)}]}{\end{enumerate}}
\def\endproofmark{$\Box$}
\newenvironment{proof}{\par{\bf Proof}:}{\endproofmark\smallskip}

\textheight=9in
\textwidth=6.5in
\topmargin=-.75in
\oddsidemargin=0.25in
\evensidemargin=0.25in

\begin{document}
\maketitle

Collaborators: 吴彦祖

% 1. 稳定匹配
\section*{1. 稳定匹配}
\textbf{笔记 4}
考虑工作岗位集合 $J = \{1, 2, 3\}$ 和求职者集合 $C = \{A, B, C\}$,其偏好如下:

\begin{center}
\begin{tabular}{c|ccc}
工作岗位 & 第一偏好 & 第二偏好 & 第三偏好 \\
\hline
1 & A & B & C \\
2 & B & A & C \\
3 & A & B & C \\
\end{tabular}
\end{center}

\begin{center}
\begin{tabular}{c|ccc}
求职者 & 第一偏好 & 第二偏好 & 第三偏好 \\
\hline
A & 2 & 1 & 3 \\
B & 1 & 3 & 2 \\
C & 1 & 2 & 3 \\
\end{tabular}
\end{center}

运行传统的提出-拒绝算法:

\begin{proof}
\textbf{第一天:}
\begin{itemize}
  \item 岗位1向A提议,岗位2向B提议,岗位3向A提议。
  \item A收到1和3的提议,A更喜欢1,暂时接受1,拒绝3。
  \item B收到2的提议,接受。
  \item C没有收到提议。
\end{itemize}

\textbf{第二天:}
\begin{itemize}
  \item 岗位3被A拒绝后,向B提议。
  \item B已被2录用,但B更喜欢1(已被A占据),次选3。B更喜欢3于2,于是B接受3,拒绝2。
  \item 2被拒绝后,下一轮向A提议。
\end{itemize}

\textbf{第三天:}
\begin{itemize}
  \item 岗位2向A提议,A已被1录用,A更喜欢1于2,拒绝2。
  \item 2向C提议,C接受。
\end{itemize}

\textbf{最终配对:}
1-A,3-B,2-C。

\textbf{共用3天。}
\end{proof}

% 2. 提出-拒绝证明
\section*{2. 提出-拒绝证明}
\textbf{笔记 4}
\begin{qparts}
\item 在算法的任何执行中,如果一名求职者在第 $i$ 天收到一个建议,那么在此后直到算法终止的每一天她都会收到一些建议。

\begin{proof}
一旦求职者收到建议,除非被更喜欢的岗位录用,否则总会有岗位向她提议(因为被拒绝的岗位会继续向下一个偏好提议)。因此直到算法结束,她每天都会收到建议。
\end{proof}

\item 在算法的任何执行中,如果一名求职者在第 $i$ 天没有收到建议,那么她在前一天 $j$($1 \leq j < i$)也没有收到建议。

\begin{proof}
岗位只会向未被录用的求职者提议。如果某天未收到建议,说明之前也未被提议。
\end{proof}

\item 在算法的任何执行中,至少有一名求职者只收到一个建议。

\begin{proof}
岗位和求职者数量相等,鸽巢原理可知,至少有一人只收到一个岗位的提议。
\end{proof}
\end{qparts}

% 3. 当裁判
\section*{3. 当裁判}
\textbf{笔记 4}
\begin{qparts}
\item 对于 $n > 1$ 的 $n$ 个工作岗位和 $n$ 个求职者,存在一个稳定匹配实例,使得在工作岗位提议的稳定匹配算法中,每个工作岗位最终都与它最不喜欢的求职者配对。

\textbf{解答:} 假。每个岗位会优先向更喜欢的求职者提议,只有被拒绝后才会向下一个提议,因此不可能所有岗位都和最不喜欢的求职者配对。

\item 在一个稳定匹配实例中,如果工作岗位 $J$ 和求职者 $C$ 各自在对方偏好列表的顶部,那么 $J$ 必须在每个稳定配对中与 $C$ 配对。

\textbf{解答:} 真。否则$J$和$C$会形成阻挡对(blocking pair),不稳定。

\item 在至少有两个工作岗位和两个求职者的稳定匹配实例中,如果工作岗位 $J$ 和求职者 $C$ 各自在对方偏好列表的底部,那么 $J$ 不能在任何稳定配对中与 $C$ 配对。

\textbf{解答:} 假。虽然互为最不喜欢,但如果其他配对会导致不稳定,$J$和$C$仍可能被配对。

\item 对于每个 $n > 1$,存在一个 $n$ 个工作岗位和 $n$ 个求职者的稳定匹配实例,其中存在一个不稳定的配对,所有未匹配的工作岗位-求职者对都是流氓配对或配对。

\textbf{解答:} 假。稳定匹配定义下不存在不稳定配对。
\end{qparts}

% 4. 稳定匹配 III
\section*{4. 稳定匹配 III}
\textbf{笔记 4}
\begin{qparts}
\item 真或假?
\begin{enumerate}[{(}i{)}]
\item 如果一名求职者在某一天意外拒绝了她更喜欢的某项工作,那么算法仍然总是以匹配结束。

\textbf{解答:} 真。算法每轮都会有岗位提议,最终所有人都会被匹配,只是结果可能不是最优。

\item 提出-拒绝算法从不产生求职者最优的匹配。

\textbf{解答:} 假。有时岗位最优和求职者最优可能重合。

\item 如果同一项工作在每个求职者的偏好列表中排在最后,那么该工作必须最终与它最不喜欢的求职者配对。

\textbf{解答:} 假。该工作可能与倒数第二喜欢的求职者配对,具体取决于其他人的选择和拒绝顺序。
\end{enumerate}

\item 构造一个示例,说明对于任意 $n \geq 2$,存在一个稳定匹配实例,使用"随意拒绝"权力对每个求职者都有帮助。

\begin{proof}
以 $n=3$ 为例:

岗位偏好:
\begin{tabular}{c|ccc}
岗位 & 第一 & 第二 & 第三 \\
\hline
1 & A & B & C \\
2 & B & C & A \\
3 & C & A & B \\
\end{tabular}

求职者偏好:
\begin{tabular}{c|ccc}
求职者 & 第一 & 第二 & 第三 \\
\hline
A & 2 & 3 & 1 \\
B & 3 & 1 & 2 \\
C & 1 & 2 & 3 \\
\end{tabular}

在岗位提议算法下,A会被1录用,B被2录用,C被3录用。若A拥有随意拒绝权,拒绝1的提议,最终A可与2配对,B与3,C与1,所有求职者都获得了更高偏好的岗位。

对于更大 $n$,可类似构造循环偏好实例。
\end{proof}
\end{qparts}

\end{document}
