% Search for all the places that say "PUT SOMETHING HERE".

\documentclass[11pt]{article}
\usepackage{amsmath,textcomp,amssymb,geometry,graphicx,enumerate}
\usepackage{xeCJK}


\def\Name{JayZhao}  % Your name
\def\SID{0}  % Your student ID number-
\def\Homework{0A} % Number of Homework
\def\Session{Spring 2024}

\title{CS70--Spring 2024 --- Discussion \Homework \  Solutions}
\author{\Name, SID \SID}
\markboth{CS70--\Session\  Discussion \Homework\ \Name}{CS70--\Session\ Discussion \Homework\ \Name}
\pagestyle{myheadings}
\date{\today}

\newenvironment{qparts}{\begin{enumerate}[{(}a{)}]}{\end{enumerate}}
\def\endproofmark{$\Box$}
\newenvironment{proof}{\par{\bf Proof}:}{\endproofmark\smallskip}

\textheight=9in
\textwidth=6.5in
\topmargin=-.75in
\oddsidemargin=0.25in
\evensidemargin=0.25in


\begin{document}
\maketitle

Collaborators: 吴彦祖

\section*{1. 自然归纳法应用于不等式}
\textbf{笔记 3} 证明:如果 $n \in \mathbb{N}$ 且 $x > 0$,则 $(1 + x)^n \geq 1 + n x$。

\begin{proof}
\textbf{基础情形}:当 $n=1$ 时,$(1+x)^1 = 1+x \geq 1+1\cdot x = 1+x$,成立。

\textbf{归纳假设}:假设对 $n=k$ 时成立,即 $(1+x)^k \geq 1+kx$。

\textbf{归纳步骤}:考虑 $n=k+1$,
$$(1+x)^{k+1} = (1+x)^k (1+x) \geq (1+kx)(1+x) = 1 + (k+1)x + kx^2 \geq 1 + (k+1)x$$
因为 $x>0$,$kx^2 \geq 0$,所以不等式成立。

因此,对所有 $n \in \mathbb{N}$,$(1+x)^n \geq 1+nx$。
\end{proof}

\section*{2. 加强证明}
\textbf{笔记 3} 假设序列 $a_1, a_2, \ldots$ 由 $a_1 = 1$ 以及对于 $n \geq 1$ 的 $a_{n+1} = 3 a_n^2$ 定义。我们想证明:
\[
a_n \leq 3^{\left(2^n\right)}
\]
对于每个正整数 $n$。

\begin{enumerate}[(a)]
\item 假设我们想使用归纳法证明这个陈述。我们的归纳假设可以简单地是 $a_n \leq 3^{\left(2^n\right)}$ 吗?尝试用这个假设进行归纳证明,说明为什么这行不通。

\textbf{解答}:
假设归纳假设 $a_n \leq 3^{2^n}$,则 $a_{n+1} = 3 a_n^2 \leq 3 (3^{2^n})^2 = 3^{1+2\cdot 2^n} = 3^{1+2^{n+1}}$,但目标是 $3^{2^{n+1}}$,多了一个 $+1$,所以无法完成归纳。

\item 尝试改为用归纳法证明陈述 $a_n \leq 3^{\left(2^n - 1\right)}$。

\textbf{解答}:
\textbf{基础情形}:$n=1$ 时,$a_1=1 \leq 3^{2^1-1}=3^1=3$,成立。

\textbf{归纳假设}:$a_n \leq 3^{2^n-1}$。

\textbf{归纳步骤}:$a_{n+1} = 3 a_n^2 \leq 3 (3^{2^n-1})^2 = 3 \cdot 3^{2\cdot(2^n-1)} = 3^{1+2^{n+1}-2} = 3^{2^{n+1}-1}$,归纳成立。

\item 为什么 (b) 部分的假设暗示了整体的结论?

\textbf{解答}:
因为 $3^{2^n-1} < 3^{2^n}$,所以 $a_n \leq 3^{2^n-1} < 3^{2^n}$,即原命题也成立。
\end{enumerate}

\section*{3. 二进制数}
\textbf{笔记 3} 证明每个正整数 $n$ 都可以用二进制表示。换句话说,证明对于任意正整数 $n$,我们可以写成:
\[
n = c_k \cdot 2^k + c_{k-1} \cdot 2^{k-1} + \cdots + c_1 \cdot 2^1 + c_0 \cdot 2^0,
\]
其中存在某个 $k \in \mathbb{N}$ 且对于所有 $i \leq k$,$c_i \in \{0, 1\}$。

\begin{proof}
用归纳法证明:

\textbf{基础情形}:$n=1$ 时,$1=1\cdot 2^0$,成立。

\textbf{归纳假设}:假设 $1,2,\ldots,k$ 都可以二进制表示。

\textbf{归纳步骤}:$k+1$ 若为偶数,则 $k+1=2m$,$m\leq k$,$m$ 有二进制表示,则 $k+1$ 也有。若为奇数,则 $k+1=2m+1$,同理。故所有正整数都可二进制表示。
\end{proof}

\section*{4. 斐波那契数在家练习}
\textbf{笔记 3} 回想一下,斐波那契数递归定义为:
\[
F_1 = 1,\ F_2 = 1 \text{,且 } F_n = F_{n-2} + F_{n-1}。
\]
证明每个第三个斐波那契数都是偶数。例如,$F_3 = 2$ 是偶数,$F_6 = 8$ 是偶数。

\begin{proof}
我们用数学归纳法证明:对于所有正整数 $k$,$F_{3k}$ 是偶数。

\textbf{基础情形}:$k=1$ 时,$F_3=2$,是偶数,成立。

\textbf{归纳假设}:假设对于某个 $k=n$,$F_{3n}$ 是偶数,即 $F_{3n}=2m$,其中 $m$ 为整数。

\textbf{归纳步骤}:我们要证明 $F_{3(n+1)}$ 也是偶数。

根据斐波那契数列递推公式:
\[
F_{3(n+1)} = F_{3n+3} = F_{3n+2} + F_{3n+1}
\]
又有
\[
F_{3n+2} = F_{3n+1} + F_{3n}
\]
\[
F_{3n+1} = F_{3n} + F_{3n-1}
\]
将 $F_{3n+2}$ 和 $F_{3n+1}$ 展开:
\[
F_{3n+3} = (F_{3n+1} + F_{3n}) + (F_{3n} + F_{3n-1}) = 2F_{3n} + F_{3n+1} + F_{3n-1}
\]
但实际上我们只需利用模2的循环性。斐波那契数列模2的周期为3,即
\[
F_1 \equiv 1 \pmod{2},\ F_2 \equiv 1 \pmod{2},\ F_3 \equiv 0 \pmod{2},\ F_4 \equiv 1 \pmod{2},\ F_5 \equiv 1 \pmod{2},\ F_6 \equiv 0 \pmod{2},\ldots
\]
所以每隔3个数必为偶数。

\textbf{结论}:因此,对于所有正整数 $k$,$F_{3k}$ 都是偶数。
\end{proof}

\end{document}
