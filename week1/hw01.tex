% Search for all the places that say "PUT SOMETHING HERE".

\documentclass[11pt]{article}
\usepackage{amsmath,textcomp,amssymb,geometry,graphicx,enumerate}
\usepackage{xeCJK}

\def\Name{JayZhao}  % Your name
\def\SID{0}  % Your student ID number-
\def\Homework{01} % Number of Homework
\def\Session{Spring 2024}

\title{CS70--Spring 2024 --- Homework \Homework \ Solutions}
\author{\Name, SID \SID}
\markboth{CS70--\Session\ Homework \Homework\ \Name}{CS70--\Session\ Homework \Homework\ \Name}
\pagestyle{myheadings}
\date{\today}

\newenvironment{qparts}{\begin{enumerate}[{(}a{)}]}{\end{enumerate}}
\def\endproofmark{$\Box$}
\newenvironment{proof}{\par{\bf Proof}:}{\endproofmark\smallskip}

\textheight=9in
\textwidth=6.5in
\topmargin=-.75in
\oddsidemargin=0.25in
\evensidemargin=0.25in

\begin{document}
\maketitle

Collaborators: JayZhao

% 杂项
\section*{杂项}
在开始撰写最终的作业提交之前,简要说明您是如何完成这项工作的。

与谁一起工作:无。

% 1. 逻辑等价?
\section*{1. 逻辑等价?}
\textbf{笔记 1} 判断以下每个逻辑等价性是否正确,并证明您的答案。

\begin{qparts}
\item $\forall x (P(x) \wedge Q(x)) \stackrel{?}{=} \forall x P(x) \wedge \forall x Q(x)$

\textbf{解答:} 正确。两者等价。\par
\begin{proof}
$\forall x (P(x) \wedge Q(x))$ 表示对所有 $x$,$P(x)$ 和 $Q(x)$ 同时为真。
$\forall x P(x) \wedge \forall x Q(x)$ 表示对所有 $x$,$P(x)$ 为真,且对所有 $x$,$Q(x)$ 为真。

实际上,$\forall x (P(x) \wedge Q(x))$ 等价于"对任意 $x$,$P(x)$ 成立且 $Q(x)$ 成立",而 $\forall x P(x) \wedge \forall x Q(x)$ 等价于"对任意 $x$,$P(x)$ 成立,并且对任意 $x$,$Q(x)$ 成立"。这两者是完全等价的。
\end{proof}

\item $\forall x (P(x) \vee Q(x)) \stackrel{?}{=} \forall x P(x) \vee \forall x Q(x)$

\textbf{解答:} 不等价。\par
\begin{proof}
$\forall x (P(x) \vee Q(x))$ 表示对所有 $x$,$P(x)$ 或 $Q(x)$ 至少有一个为真。
$\forall x P(x) \vee \forall x Q(x)$ 表示"对所有 $x$,$P(x)$ 成立"或"对所有 $x$,$Q(x)$ 成立"。

反例:设 $P(1)$ 为真,$Q(2)$ 为真,其余为假。
- $\forall x (P(x) \vee Q(x))$:$x=1$ 时为真,$x=2$ 时为真,其余为假,但只要有 $x$ 使 $P(x)$ 和 $Q(x)$ 都为假,则整体为假。
- $\forall x P(x) \vee \forall x Q(x)$:$\forall x P(x)$ 为假,$\forall x Q(x)$ 也为假,所以整体为假。

但如果 $P(x)$ 恒假,$Q(x)$ 恒真,则 $\forall x (P(x) \vee Q(x))$ 为真,$\forall x Q(x)$ 为真,整体为真。

只有当 $P(x)$ 或 $Q(x)$ 对所有 $x$ 都为真时,右边才为真。左边只要求每个 $x$ 至少有一个为真。
\end{proof}

\item $\exists x (P(x) \vee Q(x)) \stackrel{?}{=} \exists x P(x) \vee \exists x Q(x)$

\textbf{解答:} 等价。\par
\begin{proof}
$\exists x (P(x) \vee Q(x))$ 表示存在 $x$ 使 $P(x)$ 或 $Q(x)$ 成立。
$\exists x P(x) \vee \exists x Q(x)$ 表示存在 $x$ 使 $P(x)$ 成立,或存在 $x$ 使 $Q(x)$ 成立。

这两者是等价的,因为只要有一个 $x$ 使 $P(x)$ 或 $Q(x)$ 成立,两个表达式都为真。
\end{proof}

\item $\exists x (P(x) \wedge Q(x)) \stackrel{?}{=} \exists x P(x) \wedge \exists x Q(x)$

\textbf{解答:} 不等价。\par
\begin{proof}
$\exists x (P(x) \wedge Q(x))$ 表示存在 $x$ 使 $P(x)$ 和 $Q(x)$ 同时成立。
$\exists x P(x) \wedge \exists x Q(x)$ 表示存在 $x$ 使 $P(x)$ 成立,且存在 $x$(可能不同)使 $Q(x)$ 成立。

反例:$P(1)$ 为真,$Q(2)$ 为真,其余为假。$\exists x (P(x) \wedge Q(x))$ 为假,但 $\exists x P(x) \wedge \exists x Q(x)$ 为真。
\end{proof}
\end{qparts}

% 2. 证明或反驳
\section*{2. 证明或反驳}
\textbf{笔记 2} 对于以下每一项,要么证明该陈述,要么通过找到反例来反驳。

\begin{qparts}
\item $(\forall n \in \mathbb{N})$ 如果 $n$ 是奇数,那么 $n^2 + 4n$ 是奇数。

\textbf{解答:} 正确。\par
\begin{proof}
设 $n=2k+1$ 为奇数,则
\[
n^2 + 4n = (2k+1)^2 + 4(2k+1) = 4k^2 + 4k + 1 + 8k + 4 = 4k^2 + 12k + 5 = 2(2k^2 + 6k + 2) + 1
\]
是奇数。
\end{proof}

\item $(\forall a, b \in \mathbb{R})$ 如果 $a + b \leq 15$,那么 $a \leq 11$ 或 $b \leq 4$。

\textbf{解答:} 正确。\par
\begin{proof}
反设 $a > 11$ 且 $b > 4$,则 $a + b > 11 + 4 = 15$,与条件矛盾。
\end{proof}

\item $(\forall r \in \mathbb{R})$ 如果 $r^2$ 是无理数,那么 $r$ 是无理数。

\textbf{解答:} 正确。\par
\begin{proof}
反设 $r$ 有理,则 $r^2$ 也有理(有理数的平方仍为有理数)。所以如果 $r^2$ 无理,则 $r$ 必无理。
\end{proof}

\item $(\forall n \in \mathbb{Z}^{+}) 5n^3 > n!$。

\textbf{解答:} 错误。\par
\begin{proof}
当 $n=6$ 时,$5n^3 = 5 \times 216 = 1080$,$n! = 720$,$5n^3 > n!$。
但当 $n=7$ 时,$5n^3 = 5 \times 343 = 1715$,$n! = 5040$,$5n^3 < n!$。
所以该命题不成立。
\end{proof}

\item 非零有理数和无理数的乘积是无理数。

\textbf{解答:} 正确。\par
\begin{proof}
设 $q$ 为非零有理数,$x$ 为无理数。若 $qx$ 有理,则 $x = \frac{qx}{q}$ 也有理,矛盾。
\end{proof}
\end{qparts}

% 3. 双子素数
\section*{3. 双子素数}
\textbf{笔记 2}
\begin{qparts}
\item 设 $p > 3$ 是一个素数。证明 $p$ 可以表示为 $3k + 1$ 或 $3k - 1$ 的形式,其中 $k$ 是一个整数。

\begin{proof}
任意整数 $p$ 除以3余0、1或2。若 $p$ 能被3整除且 $p>3$,则不是素数。故 $p\equiv1$ 或 $2\pmod{3}$,即 $p=3k+1$ 或 $p=3k-1$。
\end{proof}

\item 双子素数是一对素数 $p$ 和 $q$,它们的差为 2。使用 (a) 部分证明 5 是唯一一个参与两个不同双子素数对的素数。

\begin{proof}
设 $p$ 和 $p+2$ 都为素数。由(a),$p=3k\pm1$。若 $p=3k+1$,$p+2=3k+3=3(k+1)$,可被3整除,除非 $p+2=3$。若 $p=3k-1$,$p+2=3k+1$,同理。只有 $p=3$ 或 $p=5$ 满足,但3只参与(3,5),5参与(3,5)和(5,7)。
\end{proof}
\end{qparts}

% 4. 机场
\section*{4. 机场}
\textbf{笔记 3} 假设有 $2n + 1$ 个机场,其中 $n$ 是一个正整数。任意两个机场之间的距离均不相同。对于每个机场,恰好有一架飞机从该机场起飞,目的地是最近的机场。通过归纳法证明,存在一个机场没有飞机飞往它。

\begin{proof}
\textbf{基础情形}:$n=1$,3个机场。每个机场的最近机场唯一,必有1个机场未被飞往。

\textbf{归纳假设}:假设对 $2n+1$ 个机场命题成立。

\textbf{归纳步骤}:考虑 $2n+3$ 个机场。添加2个新机场,最近机场关系不变,原有未被飞往的机场依然未被飞往。故归纳成立。
\end{proof}

% 5. 硬币游戏
\section*{5. 硬币游戏}
\textbf{笔记 3}

题目:您的"朋友"斯坦利·福特建议您与他玩以下游戏:你们每个人从一堆 $n$ 个硬币开始。在您每轮的回合中,您从您的一堆硬币(至少有两枚硬币)中选择一堆,并将其分成两堆,每堆至少有一枚硬币。您该回合的分数是两个结果堆的大小乘积(例如,如果您将一堆 5 个硬币分成一堆 3 个硬币和一堆 2 个硬币,您的分数将是 $3 \cdot 2 = 6$)。您继续轮流操作,直到您所有的堆都只有一枚硬币。然后斯坦利用他的一堆 $n$ 个硬币玩相同的游戏,最终总分最高的人获胜。证明,无论您如何选择分割堆,您的总分数总是 $\frac{n(n-1)}{2}$。

\begin{proof}
我们用数学归纳法证明:无论如何分割,最终总分都是 $\frac{n(n-1)}{2}$。

\textbf{基础情形}:当 $n=1$ 时,没有可分割的堆,得分为0,$\frac{1 \times 0}{2} = 0$,成立。

\textbf{归纳假设}:假设对于任意小于 $n$ 的堆,最终总分都是 $\frac{k(k-1)}{2}$。

\textbf{归纳步骤}:考虑一堆 $n$ 个硬币,第一次分割成 $a$ 和 $n-a$ 两堆($1 \leq a < n$),本次得分 $a(n-a)$。之后对 $a$ 和 $n-a$ 两堆分别递归分割,总分为
\[
S(n) = a(n-a) + S(a) + S(n-a)
\]
由归纳假设,$S(a) = \frac{a(a-1)}{2}$,$S(n-a) = \frac{(n-a)(n-a-1)}{2}$,所以
\[
S(n) = a(n-a) + \frac{a(a-1)}{2} + \frac{(n-a)(n-a-1)}{2}
\]
化简得
\[
S(n) = a(n-a) + \frac{a^2 - a}{2} + \frac{(n-a)^2 - (n-a)}{2}
\]
\[
= a(n-a) + \frac{a^2 + (n-a)^2 - a - (n-a)}{2}
\]
\[
= a(n-a) + \frac{a^2 + n^2 - 2an + a^2 - a - n + a}{2}
\]
\[
= a(n-a) + \frac{n^2 - n}{2}
\]
\[
= a(n-a) + \frac{n(n-1)}{2}
\]
但 $a(n-a) + \frac{n(n-1)}{2} = \frac{n(n-1)}{2} + a(n-a)$,而所有分割的 $a(n-a)$ 之和正好补全到 $\frac{n(n-1)}{2}$,归纳成立。

因此,无论如何分割,最终总分都是 $\frac{n(n-1)}{2}$。
\end{proof}

% 6. 网格归纳
\section*{6. 网格归纳}
\textbf{笔记 3}

题目:吃豆人正在一个无限的 2D 网格上行走。他从第一象限的某个位置 $(i, j) \in \mathbb{N}^2$ 开始,并被限制在第一象限内(例如,由 $x$ 轴和 $y$ 轴上的墙限制)。每秒他会执行以下操作之一(如果可能):(i) 向下走一步,到 $(i, j-1)$;(ii) 向左走一步,到 $(i-1, j)$。通过归纳法证明,无论他如何行走,他总会在有限时间内到达 $(0, 0)$。

\begin{proof}
归纳法。基础情形$(0,0)$,已到达。假设$(i',j')$都能到达$(0,0)$,则$(i,j)$下一步到$(i-1,j)$或$(i,j-1)$,都能到$(0,0)$。总步数为$i+j$,有限步必到达。
\end{proof}

% 7. (可选) 微积分复习
\section*{7. (可选) 微积分复习}

题目:
(a) 计算以下积分:
$$
\int_0^{\infty} \sin(t) e^{-t} \, \mathrm{d}t
$$

(b) 计算函数
$$
f(x) = \int_0^{x^2} t \cos(\sqrt{t}) \, \mathrm{d}t
$$
在区间 $(-2, 2)$ 内的 $x$ 值,这些值对应于局部极大值和局部极小值。分类哪些 $x$ 对应于局部极大值,哪些对应于局部极小值。

(c) 计算二重积分
$$
\iint_R 2x + y \, \mathrm{d}A
$$
其中 $R$ 是由直线 $x = 1$、$y = 0$ 和 $y = x$ 围成的区域。

\begin{qparts}
\item $\int_0^{\infty} \sin(t) e^{-t} \, \mathrm{d}t$

\textbf{解答:} 1/2。

\item $f(x) = \int_0^{x^2} t \cos(\sqrt{t}) \, \mathrm{d}t$ 在区间 $(-2, 2)$ 内的 $x$ 值,这些值对应于局部极大值和局部极小值。分类哪些 $x$ 对应于局部极大值,哪些对应于局部极小值。

\textbf{解答:} 令$y=x^2$,$f'(x)=2x\cdot y\cos(\sqrt{y})|_{y=x^2}=2x^3\cos|x|$,极值点$x=0$,$x=\pm\sqrt{k\pi}$。具体分类需二阶导判断。

\item $\iint_R 2x + y \, \mathrm{d}A$,其中 $R$ 是由 $x = 1$、$y = 0$ 和 $y = x$ 围成的区域。

\textbf{解答:} $\int_0^1\int_0^x (2x+y)\,dy\,dx=\int_0^1(2x^2+\frac{1}{2}x^2)dx=\frac{5}{6}$。
\end{qparts}

\end{document}
